\documentclass[extra]{gji}
%~ %\documentclass[extra, referee]{gji}

\usepackage[utf8]{inputenc}
\usepackage{timet}
\usepackage{amsmath}
\usepackage{graphicx}
\usepackage{todonotes} % to make annotations on margins

\usepackage{url}
\usepackage[pdftex,colorlinks=true]{hyperref}
\hypersetup{
    allcolors=blue,
}


\begin{document}

\title[Variable Density Tesseroids]{
    Variable Density Tesseroids: Gravity fields calculation in spherical coordinates using variable densities in depth
}
\author[S.R. Soler, L. Uieda and M.E. Gimenez]{
    Santiado R. Soler$^{1,2}$, Leonardo Uieda$^3$ and Mario E. Gimenez$^{1,2}$ \\
    $^1$CONICET, Argentina. e-mail: santiago.r.soler@gmail.com\\
    $^2$Instituto Geofísico Sismológico Volponi, Universidad Nacional de San Juan, Argentina\\
    $^3$Universidade do Estado do Rio de Janeiro, Rio de Janeiro, Brazil
    }


\maketitle

\begin{summary}
Summary of this paper 
\end{summary}

\begin{keywords}
up to six keywords from the list
\end{keywords}



\section{Introduction}

Lithosphere density variation with depth has been studied for almost a century. 
\citet{Athy1930} obtained a decreasing exponential relation between both quantities by studying shale samples.
In the following years, other density functions have been proposed for different rock types \citep[e.g.,][]{Maxant1980, Rao1986, Rao1993, Rao1994}.
Furthermore, the density variation with depth has been taken into account in forward and inversion gravity models, mostly applied to grabens and basins \citep{Cordell1973, Rao1986, Cowie1990, Rao1993, Rao1994, Zhang2001, Welford2010}.

These forward gravity models have been developed for two or three dimensional bodies in cartesian coordinates that properly fit local scales applications.
Nevertheless, the latest satellite missions have provided us gravity measurements with global coverage, which allows geologists and geophysics to perform modelling and interpretation in regional scales.
This raises the importance of building forward models that reassemble the gravity anomalies for such scales.

The main issue that should be overcome is taking into account the curvature of the Earth, thus the forward model must be defined in spherical coordinates.
A common way to achieve this is to discretize the Earth in spherical prisms known as tesseroids, which are defined by pairs of latitude, longitude and radius boundaries \todo{Insertar figura de tesseroides de leo}.
Given an arbitrary tesseroid, calculating the gravity fields on any external point involves the resolution of volume integrals that are generally approximated by numerical computations.
The literature offers two approaches: one involves Taylor series expansion \citep{Heck2007, Grombein2013} while the other makes use of Gauss-Legendre Quadrature (GLQ) \citep{Asgharzadeh2007, Uieda2016, Uieda2017}. The later consists in approximating the integral by a weighted sum of the effect of point masses located at scaled nodes of the Legendre polynomials.

In order to develop a forward model that computes the gravity field generated by any tesseroid with arbitrary variable density, the Taylor series expansion is not well suited, because it would need to derive the series expansion terms for each density function. On the other hand, the GLQ allows us to calculate them for any density function without any other information but the function itself.

\citet{Uieda2016} developed a forward gravity model based on tesseroids with homogeneous densities using the GLQ approximation method.
It has been originally implemented in command line programs written in C programming language. Although, it has been ultimately ported to Python and included in the open-source library Fatiando a Terra \citep{Uieda2013} for geophysical modelling and inversion.

It's well known that the GLQ method becomes less accurate when the computation point is closer to the tesseroid \citep{Ku1977} or for smaller GLQ order. 
Although the model developed by \citet{Uieda2016} uses a second order GLQ, it ensures the accuracy implementing a modified version of the adaptative discretization of \citet{Li2011}.
It consist in splitting the tesseroid into smaller ones when a certain distance-size ratio ($D$) is greater than a predefined value that controls the precision of the computation.  \citet{Uieda2016} have also objectively obtained standard values of $D$ for gravity potential, gradients and tensor components comparing the fields generated by a spherical shell, which constitutes a special case of the volume integrals that has analytical solution \citep{LaFehr1991, Mikuska2006, Grombein2013}\todo{citar mas??}.
The modifications made by \citet{Uieda2016} to the original adaptative discretization method \citep{Li2011} consist in:
(1) a faster calculation of the distance from the computation point to the tesseroid and 
(2) the application of a stack based algorithm that speeds up the computation and gives more control over the recursion step.

We have enhanced the later implementation with a new forward gravity model that allows us to compute the gravity fields generated by any tesseroid with an arbitrary density function on any external point.
It's written in Python and Cython languages, obtaining an easy to use library that runs precompiled code for the more time consuming functions.

In order to ensure the accuracy of the numerical approximation, we have compared its results with a spherical shell with linear and exponential density functions, similar to the test performed by \cite{Uieda2016}. Although in this case, the analytical expressions for both density functions had to be derived.

It also makes use of some previously existing classes and functions from Fatiando a Terra, what saved us time and allowed us to focus on the new enhancements. Furthermore, we have the intention to include it into a future release of the Fatiando a Terra library in order to make its distribution and maintenance easier.

In the following sections we will describe how the new algorithm works, its theoretical background and obtain the distance-size ratio values needed to get a good accuracy in the computation.

%%%%%%%%%%%%%%%%%%%%%%%%%%%%%%%%%%%%%%%%%%%%%%%%%%%%%%%%%%%%%%%%%%%%%%%%%%%%%%

\section{Theory}

The spherical prisms known as tesseroids are mass elements defined in a geocentric spherical coordinate system bounded by a pair of parallels, a pair of  meridians, and two concentric spherical surfaces.
We define an external computation point $P(r, \phi, \lambda)$ where the gravity fields generated by the tesseroid are going to be calculated with respect to the local north oriented cartesian coordinate system located at $P$.
In the special case of homogeneous density, the gravity potential, gradient and Marussi tensor components can be calculated through the integrals obtained by \citet{Grombein2013} \citep[see also][]{Uieda2016}.

In our case, we will suppose that the tesseroid has an arbitrary variable density in depth, this means that only depends on the radius spherical coordinate. Thus, the integrals for the gravity fields are slightly modified.

\begin{equation}
    V(r,\phi,\lambda) = G
    \int\limits_{\lambda_1}^{\lambda_2}
    \int\limits_{\phi_1}^{\phi_2}
    \int\limits_{r_1}^{r_2}
    \frac{\rho(r')}{\ell} \kappa \,  dr' d\phi' d\lambda',
\label{eq:tesseroid-pot}
\end{equation}
\begin{equation}
    g_{\alpha}(r,\phi,\lambda) = G
    \int\limits_{\lambda_1}^{\lambda_2}
    \int\limits_{\phi_1}^{\phi_2}
    \int\limits_{r_1}^{r_2}
    \rho(r') \frac{\Delta_\alpha}{\ell^3}
    \kappa \, dr' d\phi' d\lambda',
\label{eq:tesseroid-grav}
\end{equation}
\begin{equation}
    g_{\alpha\beta}(r,\phi,\lambda) = G
    \int\limits_{\lambda_1}^{\lambda_2}
    \int\limits_{\phi_1}^{\phi_2}
    \int\limits_{r_1}^{r_2}
    \rho(r') I_{\alpha\beta} \, \kappa \, dr' d\phi' d\lambda' ,
    \label{eq:tesseroid-tensor}
\end{equation}
\noindent where
\begin{equation}
    I_{\alpha\beta} =
    \left(
        \frac{3\Delta_{\alpha} \Delta_{\beta}}{\ell^5} -
        \frac{\delta_{\alpha\beta}}{\ell^3}
    \right) ,
    \label{eq:tesseroid-tensor-kernel}
\end{equation}

\noindent $\alpha, \beta \in \{x, y, z\}$,
$\rho(r')$ is the density function that depends on the radius coordinate,
$\delta_{\alpha\beta}$ is Kronecker's delta,
$G = 6.674\times10^{-11}\, \text{m$^3$kg$^{-1}$s$^{-1}$}$ is the gravitational constant and
\begin{equation}
    \Delta_x = r'[\cos\phi\sin\phi' - \sin\phi\cos\phi'
               \cos(\lambda' - \lambda)],
\end{equation}
\begin{equation}
    \Delta_y = r' \cos \phi' \sin(\lambda' - \lambda),
\end{equation}
\begin{equation}
    \Delta_z = r' \cos \psi - r,
\end{equation}
\begin{equation}
    \kappa = {r'}^2 \cos \phi',
\end{equation}
\begin{equation}
    \ell = \sqrt{{r'}^2 + r^2 - 2 r' r \cos \psi},
\end{equation}
\begin{equation}
    \cos\psi = \sin\phi\sin\phi' + \cos\phi\cos\phi'
                 \cos(\lambda' - \lambda).
\end{equation}


According to \citet[p.~390]{Hildebrand1987}, we can approximate any integral in the interval $[-1, 1]$ using a $N$th order GLQ, i.e. by a weighted sum of the integration kernel evaluated on the roots of the $N$th Legendre polynomial:

\begin{equation}
    \int\limits_{-1}^1 f(x) dx \approx \frac{b-a}{2} \sum_{i=1}^N W_i f(x_i),
\end{equation}

\noindent where $x_i$ are the roots of the $N$th Legendre polynomial $P_N(x)$ and $W_i$ are the weights of each term:

\begin{equation}
    W_i = \frac{2}{(1-x_i^2)[P_N^\prime(x_i)]^2},
\end{equation}

\begin{equation}
    P_N(x_i) = 0, \quad \forall i = {1,\dots,N}.
\end{equation}

In case of an integration defined on a arbitrary interval, e.g. $[a,b]$, we can scale the nodes and perform a similar approximation:

\begin{equation}
    \int\limits_a^b f(x) dx \approx \frac{b-a}{2} \sum_{i=1}^N W_i f(x_i^s),
\end{equation}

\noindent where $x_i^s$ is the scaled $x_i$ root in the $[a,b]$ interval, also called nodes of the quadrature:

\begin{equation}
    x_i^s = \frac{b-a}{2} x_i + \frac{b+a}{2}.
\end{equation}

For example, if we use a second order GLQ, the roots of the $P_2(x)$ Legendre polynomial and its corresponding weights are $x_i = \pm 0.577350269$ and $W_i = 1$, respectivetly.

Our intention is to use GLQ to approximate the volume integrals from equations \ref{eq:tesseroid-pot}-\ref{eq:tesseroid-tensor}. 
When tesseroids have variable densities, we can write the integral kernels as the product of a certain function $g$ and the density function:

\begin{equation}
    f(r', \phi', \lambda') = \rho(r') g(r', \phi', \lambda'),
\end{equation}

\noindent and then apply the quadrature to every integral corresponding to each spherical coordinate, obtaining a similar expression to the one shown by \citet{Asgharzadeh2007} and \citet{Uieda2016}:

\iftwocol{
\begin{equation}
\begin{split}
    \iiint\limits_\Omega \rho(r') g(r', \phi', \lambda') d\Omega \approx& \\
    A 
    \sum\limits_{i=1}^{N^r}
    \sum\limits_{j=1}^{N^\phi}
    \sum\limits_{k=1}^{N^\lambda}
    & W_i^r W_j^\phi W_k^\lambda \rho(r_i) g(r_i, \phi_j, \lambda_k),
\end{split}
\label{eq:glq-var-dens}
\end{equation}
}
{
\begin{equation}
    \iiint\limits_\Omega \rho(r') g(r', \phi', \lambda') d\Omega \approx
    A 
    \sum\limits_{i=1}^{N^r}
    \sum\limits_{j=1}^{N^\phi}
    \sum\limits_{k=1}^{N^\lambda}
    W_i^r W_j^\phi W_k^\lambda \rho(r_i) g(r_i, \phi_j, \lambda_k),
\label{eq:glq-var-dens}
\end{equation}
}

\noindent where

\begin{equation}
    A = 
    \frac{(\lambda_2 - \lambda_1)(\phi_2 - \phi_1)(r_2 - r_1)}{8}.
\end{equation}

From equation \ref{eq:glq-var-dens} we can observe that the GLQ approximates the gravity fields of a tesseroid with variable density as the ones generated by point masses located at the nodes of the Legendre polynomials.
Furthermore, the information of the density function is summarised as the values it assumes in those nodes.
Although this may sound discouraging, we intent to prove otherwise and show that this method, along with a well fitted discretization algorithm, approximates the gravity fields with good accuracy and precision.\todo{no se si poner esta ultima oracion}

%%%%%%%%%%%%%%%%%%%%%%%%%%%%%%%%%%%%%%%%%%%%%%%%%%%%%%%%%%%%%%%%%%%%%%%%%%%%%%

\subsection{Analytical Solutions for Spherical Shell}

In order to test the accuracy of the GLQ approximation, \citet{Uieda2016} compared the computed gravity fields with the ones generated by a spherical shell, the exceptional case with analytical solution \citep{Misuska2006,}.

In case of variable density tesseroids, the comparison must obviously be made with a variable density spherical shell, whose analytical solution will depend on the chosen density function.
Moreover, the literature does not develop the analytical solution


%%%%%%%%%%%%%%%%%%%%%%%%%%%%%%%%%%%%%%%%%%%%%%%%%%%%%%%%%%%%%%%%%%%%%%%%%%%%%%%
\section{Acknowledgments}

We are indebted to the developers and maintainers of the open-source
software without which this work would not have been possible.

%%%%%%%%%%%%%%%%%%%%%%%%%%%%%%%%%%%%%%%%%%%%%%%%%%%%%%%%%%%%%%%%%%%%%%%%%%%%%%%

\bibliographystyle{gji}
\bibliography{bibtex/references}

\end{document}
